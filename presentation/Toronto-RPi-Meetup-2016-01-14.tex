%
% Toronto-RPi-Meetup-2016-01-14.tex
% 
% Copyright (c), 2016 Syed Faisal Akber
%
% All references will be noted in a file called references.txt
%
\documentclass[slidestop,usepdftitle=false,14pt,table]{beamer}
\usepackage[accumulated]{beamerseminar}
\usepackage{beamertexpower}
\usepackage{hyperref}
\usepackage{graphicx}
\usepackage{color}
\usepackage{colortbl}
\usepackage{transparent}

\usepackage[bars]{beamerthemetree}

\usetheme{Madrid}

% Configure Beamer to use the VMware colours
\setbeamercolor{structure}{fg=gray}
\setbeamercolor{example text}{fg=green}
\setbeamercolor{normal text}{fg=white,bg=black}

% Footline chages and remove navigation symbols
\setbeamertemplate{footline}[page number]{}
\beamertemplatenavigationsymbolsempty

% Define title, author and other important information
\title[Raspberry Pi GPIO]
      {Using the GPIO interface on the Raspberry Pi}
%\subtitle{}
\author[S. F. Akber]{Syed Faisal Akber}

% PDF metadata setup
\hypersetup{pdftitle=
  {Using the GPIO interface on the Raspberry Pi},
  pdfauthor={Syed Faisal Akber}}
% Since we're using the background image for the template we don't need to 
% paste the logo everywhere.
%%%\logo{\includegraphics[height=0.8cm]{vmware.eps}}

% Date of the presentation
\date{2016-01-14}

\begin{document}
% Title page
% Set the background to the title page image
{
%  \setbeamertemplate{background}{\includegraphics[width=\paperwidth,height=\paperheight,keepaspectratio]{template-title0.jpg}}
  \begin{frame}
    \maketitle
  \end{frame}
}

% Insert table of contents (a.k.a. Agenda based upon sections and subsections)
{
%  \setbeamertemplate{background}{\includegraphics[width=\paperwidth,height=\paperheight,keepaspectratio]{template-sectiontitle1.jpg}}
\begin{frame}
  %  \newslide
  \tableofcontents
\end{frame}
}
%% Example flow of a slide/frame
%% \section{A list environment}
%% \frame{
%% \begin{slide}
%% \centerslidesfalse
%% \frametitle{A list environment}

%% \pause
%% \stepwise
%% {%
%%   \begin{description}
%%   \item[foo.] \step{bar.}%%   \step{\item[baz.]} \step{qux.}
%%   \end{description}
%%   }

%% \end{slide}
%%}

% Set the default background for each slide.
%\setbeamertemplate{background}{\includegraphics[width=\paperwidth,height=\paperheight,keepaspectratio]{template-normal.jpg}}

% Start content here
\section{Introduction}
\begin{frame}
\frametitle{Introduction}
\begin{itemize}
\item Raspberry Pi has many interfaces
\item Some we are already familiar with
\item Today we'll look at P1 (GPIO Interface)
\item Its basic usage and pointers to more detail
\end{itemize}
\end{frame}

\section{P1 - GPIO Header}
\begin{frame}
\frametitle{GPIO Pinouts}
\begin{itemize}
\item **** add pics here
\item talk about pintout.xyz ****
\end{itemize}
\end{frame}

\section{Simple I/O}
\begin{frame}
\frametitle{Simple Output}
**** simple outputs
**** add example with LED, python and C
\end{frame}

\begin{frame}
\frametitle{Simple Input}
**** simple inputs
**** use switch 
**** talk about pull-up and pull-down resistors
**** talk about it being already included
\end{frame}

\section{Pulse-Width Modulation (PWM)}
\begin{frame}
\frametitle{Pulse-Width Modulation (PWM)}
**** uses of PWM
**** elude to pifm
\end{frame}

\section{Serial Communications}
\begin{frame}
\frametitle{Serial Communications}


\end{frame}



\section{Serial Peripheral Interface (SPI)}
\begin{frame}
\frametitle{Serial Peripheral Interface (SPI)}


\end{frame}


\section{$I^2C$}
\begin{frame}
\frametitle{$I^2C$}


\end{frame}




\section{Other Topics}
\begin{frame}
\frametitle{Scratch}
**** add links to GPIO version
\end{frame}

\begin{frame}
\frametitle{What to do when you're running out of pins!}
**** Multiplexing
**** specialized drivers
**** daisy-chaining
**** guzunty
\end{frame}

\begin{frame}
\frametitle{Common Peripherals}
\begin{itemize}
\item **** list all items **** make frame multi-page
\end{itemize}
\end{frame}

% Conclusions
\section{Summary}
\begin{frame}
  \frametitle{Summary}
  \begin{itemize}
  \item Interfaces
  \item Pinouts
  \item Interfacing techniques
  \item ...
  \end{itemize}
\end{frame}

% Questions slide
{
%\setbeamertemplate{background}{\includegraphics[width=\paperwidth,height=\paperheight,keepaspectratio]{template-quote2.jpg}}
\setbeamercolor{structure}{fg=gray}
\begin{frame}
  \frametitle{Questions?}
  \centerslidestrue
\end{frame}
}

% References slide
\frame{
  \frametitle{References}
  \bibliographystyle{abbrvnat}
  \begin{thebibliography}{}
    \bibitem[]{RPi}
      \emph{Raspberry Pi}
      \url{http://www.raspberrypi.org/}
    \bibitem[]{RPi-GPIO-pinouts}
      \emph{GPIO: Raspberry Pi Models A and B}
      \url{https://www.raspberrypi.org/documentation/usage/gpio/}
  \end{thebibliography}
}

\end{document}
